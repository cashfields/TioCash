\documentclass[12pt,letterpaper,oneside,titlepage]{article}
\usepackage[latin1]{inputenc}
\usepackage{amsmath}
\usepackage{amsfonts}
\usepackage{amssymb}
\usepackage{graphicx}
\usepackage[letterpaper, portrait, margin=.5in]{geometry}
\usepackage{lscape}
\usepackage{tikz}
\usetikzlibrary{matrix,calc}
\usepackage{color}
\usepackage{tabto}
\usepackage{pgfplots}
\pgfplotsset{compat=1.7}
\pgfplotsset{width=7cm,compat=1.13}%




\usepackage[]{hyperref}
\hypersetup{
    %	pdftitle={Tio Cash's Sort of Primes},
    %	pdfauthor={Scott \textbf{``Cash''} Fields},
    %	pdfsubject={Tio Cash's Sort of Primes},
    %pdfkeywords={keyword1, keyword2},
    bookmarksnumbered=true,     
    bookmarksopen=true,         
    bookmarksopenlevel=1,       
    colorlinks=true,   
    allcolors=blue,        
    pdfstartview=Fit,           
    pdfpagemode=UseOutlines,    % this is the option you were lookin for
    %	pdfpagelayout=TwoPageRight
}
%%%%%%%%%%%%%%%%%%%%%%%%%%%%
%%%%%%%%%%%%%%%%%%%%%%%%%%%%
%%%%%%%%% cube pile %%%%%%%%%%%%%
%\documentclass[parskip]{scrartcl}
%\usepackage[margin=15mm,landscape]{geometry}
%\usepackage{tikz}
\usepackage{keyval}
\usepackage{ifthen}
%====================================
%emphasize vertices --> switch and emph style (e.g. thick,black)
%====================================
\makeatletter
% Standard Values for Parameters
\newcommand{\tikzcuboid@shiftx}{0}
\newcommand{\tikzcuboid@shifty}{0}
\newcommand{\tikzcuboid@dimx}{3}
\newcommand{\tikzcuboid@dimy}{3}
\newcommand{\tikzcuboid@dimz}{3}
\newcommand{\tikzcuboid@scale}{1}
\newcommand{\tikzcuboid@densityx}{1}
\newcommand{\tikzcuboid@densityy}{1}
\newcommand{\tikzcuboid@densityz}{1}
\newcommand{\tikzcuboid@rotation}{0}
\newcommand{\tikzcuboid@anglex}{0}
\newcommand{\tikzcuboid@angley}{90}
\newcommand{\tikzcuboid@anglez}{225}
\newcommand{\tikzcuboid@scalex}{1}
\newcommand{\tikzcuboid@scaley}{1}
\newcommand{\tikzcuboid@scalez}{sqrt(0.5)}
\newcommand{\tikzcuboid@linefront}{black}
\newcommand{\tikzcuboid@linetop}{black}
\newcommand{\tikzcuboid@lineright}{black}
\newcommand{\tikzcuboid@fillfront}{white}
\newcommand{\tikzcuboid@filltop}{white}
\newcommand{\tikzcuboid@fillright}{white}
\newcommand{\tikzcuboid@shaded}{N}
\newcommand{\tikzcuboid@shadecolor}{black}
\newcommand{\tikzcuboid@shadeperc}{25}
\newcommand{\tikzcuboid@emphedge}{N}
\newcommand{\tikzcuboid@emphstyle}{thick}

% Definition of Keys
\define@key{tikzcuboid}{shiftx}[\tikzcuboid@shiftx]{\renewcommand{\tikzcuboid@shiftx}{#1}}
\define@key{tikzcuboid}{shifty}[\tikzcuboid@shifty]{\renewcommand{\tikzcuboid@shifty}{#1}}
\define@key{tikzcuboid}{dimx}[\tikzcuboid@dimx]{\renewcommand{\tikzcuboid@dimx}{#1}}
\define@key{tikzcuboid}{dimy}[\tikzcuboid@dimy]{\renewcommand{\tikzcuboid@dimy}{#1}}
\define@key{tikzcuboid}{dimz}[\tikzcuboid@dimz]{\renewcommand{\tikzcuboid@dimz}{#1}}
\define@key{tikzcuboid}{scale}[\tikzcuboid@scale]{\renewcommand{\tikzcuboid@scale}{#1}}
\define@key{tikzcuboid}{densityx}[\tikzcuboid@densityx]{\renewcommand{\tikzcuboid@densityx}{#1}}
\define@key{tikzcuboid}{densityy}[\tikzcuboid@densityy]{\renewcommand{\tikzcuboid@densityy}{#1}}
\define@key{tikzcuboid}{densityz}[\tikzcuboid@densityz]{\renewcommand{\tikzcuboid@densityz}{#1}}
\define@key{tikzcuboid}{rotation}[\tikzcuboid@rotation]{\renewcommand{\tikzcuboid@rotation}{#1}}
\define@key{tikzcuboid}{anglex}[\tikzcuboid@anglex]{\renewcommand{\tikzcuboid@anglex}{#1}}
\define@key{tikzcuboid}{angley}[\tikzcuboid@angley]{\renewcommand{\tikzcuboid@angley}{#1}}
\define@key{tikzcuboid}{anglez}[\tikzcuboid@anglez]{\renewcommand{\tikzcuboid@anglez}{#1}}
\define@key{tikzcuboid}{scalex}[\tikzcuboid@scalex]{\renewcommand{\tikzcuboid@scalex}{#1}}
\define@key{tikzcuboid}{scaley}[\tikzcuboid@scaley]{\renewcommand{\tikzcuboid@scaley}{#1}}
\define@key{tikzcuboid}{scalez}[\tikzcuboid@scalez]{\renewcommand{\tikzcuboid@scalez}{#1}}
\define@key{tikzcuboid}{linefront}[\tikzcuboid@linefront]{\renewcommand{\tikzcuboid@linefront}{#1}}
\define@key{tikzcuboid}{linetop}[\tikzcuboid@linetop]{\renewcommand{\tikzcuboid@linetop}{#1}}
\define@key{tikzcuboid}{lineright}[\tikzcuboid@lineright]{\renewcommand{\tikzcuboid@lineright}{#1}}
\define@key{tikzcuboid}{fillfront}[\tikzcuboid@fillfront]{\renewcommand{\tikzcuboid@fillfront}{#1}}
\define@key{tikzcuboid}{filltop}[\tikzcuboid@filltop]{\renewcommand{\tikzcuboid@filltop}{#1}}
\define@key{tikzcuboid}{fillright}[\tikzcuboid@fillright]{\renewcommand{\tikzcuboid@fillright}{#1}}
\define@key{tikzcuboid}{shaded}[\tikzcuboid@shaded]{\renewcommand{\tikzcuboid@shaded}{#1}}
\define@key{tikzcuboid}{shadecolor}[\tikzcuboid@shadecolor]{\renewcommand{\tikzcuboid@shadecolor}{#1}}
\define@key{tikzcuboid}{shadeperc}[\tikzcuboid@shadeperc]{\renewcommand{\tikzcuboid@shadeperc}{#1}}
\define@key{tikzcuboid}{emphedge}[\tikzcuboid@emphedge]{\renewcommand{\tikzcuboid@emphedge}{#1}}
\define@key{tikzcuboid}{emphstyle}[\tikzcuboid@emphstyle]{\renewcommand{\tikzcuboid@emphstyle}{#1}}
% Commands
\newcommand{\tikzcuboid}[1]{
    \setkeys{tikzcuboid}{#1} % Process Keys passed to command
    \pgfmathsetmacro{\vectorxx}{\tikzcuboid@scalex*cos(\tikzcuboid@anglex)}
    \pgfmathsetmacro{\vectorxy}{\tikzcuboid@scalex*sin(\tikzcuboid@anglex)}
    \pgfmathsetmacro{\vectoryx}{\tikzcuboid@scaley*cos(\tikzcuboid@angley)}
    \pgfmathsetmacro{\vectoryy}{\tikzcuboid@scaley*sin(\tikzcuboid@angley)}
    \pgfmathsetmacro{\vectorzx}{\tikzcuboid@scalez*cos(\tikzcuboid@anglez)}
    \pgfmathsetmacro{\vectorzy}{\tikzcuboid@scalez*sin(\tikzcuboid@anglez)}
    \begin{scope}[xshift=\tikzcuboid@shiftx, yshift=\tikzcuboid@shifty, scale=\tikzcuboid@scale, rotate=\tikzcuboid@rotation, x={(\vectorxx,\vectorxy)}, y={(\vectoryx,\vectoryy)}, z={(\vectorzx,\vectorzy)}]
        \pgfmathsetmacro{\steppingx}{1/\tikzcuboid@densityx}
        \pgfmathsetmacro{\steppingy}{1/\tikzcuboid@densityy}
        \pgfmathsetmacro{\steppingz}{1/\tikzcuboid@densityz}
        \newcommand{\dimx}{\tikzcuboid@dimx}
        \newcommand{\dimy}{\tikzcuboid@dimy}
        \newcommand{\dimz}{\tikzcuboid@dimz}
        \pgfmathsetmacro{\secondx}{2*\steppingx}
        \pgfmathsetmacro{\secondy}{2*\steppingy}
        \pgfmathsetmacro{\secondz}{2*\steppingz}
        \foreach \x in {\steppingx,\secondx,...,\dimx}
        {   \foreach \y in {\steppingy,\secondy,...,\dimy}
            {   \pgfmathsetmacro{\lowx}{(\x-\steppingx)}
                \pgfmathsetmacro{\lowy}{(\y-\steppingy)}
                \filldraw[fill=\tikzcuboid@fillfront,draw=\tikzcuboid@linefront] (\lowx,\lowy,\dimz) -- (\lowx,\y,\dimz) -- (\x,\y,\dimz) -- (\x,\lowy,\dimz) -- cycle;
                
            }
        }
        \foreach \x in {\steppingx,\secondx,...,\dimx}
        {   \foreach \z in {\steppingz,\secondz,...,\dimz}
            {   \pgfmathsetmacro{\lowx}{(\x-\steppingx)}
                \pgfmathsetmacro{\lowz}{(\z-\steppingz)}
                \filldraw[fill=\tikzcuboid@filltop,draw=\tikzcuboid@linetop] (\lowx,\dimy,\lowz) -- (\lowx,\dimy,\z) -- (\x,\dimy,\z) -- (\x,\dimy,\lowz) -- cycle;
            }
        }
        \foreach \y in {\steppingy,\secondy,...,\dimy}
        {   \foreach \z in {\steppingz,\secondz,...,\dimz}
            {   \pgfmathsetmacro{\lowy}{(\y-\steppingy)}
                \pgfmathsetmacro{\lowz}{(\z-\steppingz)}
                \filldraw[fill=\tikzcuboid@fillright,draw=\tikzcuboid@lineright] (\dimx,\lowy,\lowz) -- (\dimx,\lowy,\z) -- (\dimx,\y,\z) -- (\dimx,\y,\lowz) -- cycle;
            }
        }
        \ifthenelse{\equal{\tikzcuboid@emphedge}{Y}}%
        {\draw[\tikzcuboid@emphstyle](0,\dimy,0) -- (\dimx,\dimy,0) -- (\dimx,\dimy,\dimz) -- (0,\dimy,\dimz) -- cycle;%
            \draw[\tikzcuboid@emphstyle] (0,0,\dimz) -- (0,\dimy,\dimz) -- (\dimx,\dimy,\dimz) -- (\dimx,0,\dimz) -- cycle;%
            \draw[\tikzcuboid@emphstyle](\dimx,0,0) -- (\dimx,\dimy,0) -- (\dimx,\dimy,\dimz) -- (\dimx,0,\dimz) -- cycle;%
        }%
        {}
    \end{scope}
}
\makeatother

%%%%%%%%%%%%%%%%%%%%%%%%%%%%
%%%%%%%%%%%%%%%%%%%%%%%%%%%%
% https://tex.stackexchange.com/questions/103123/links-do-not-lead-to-right-pages

\begin{document}
%    \author{Scott \textbf{``Cash''} Fields}
%    \title{Tio Cash's Sort of Primes}
%    \maketitle
%    
    
    %%%%%%%%%%%%%%%%%%%%%%%%%%%%%%%%%%%%%%%%%%%%%
    %links in toc will not jump to abstract
    %https://tex.stackexchange.com/questions/103123/links-do-not-lead-to-right-pages
    %http://latex.org/forum/viewtopic.php?t=1205
    %dedication
    %http://gradschool.unc.edu/academics/thesis-diss/guide/ordercomponents.html
    
    
    %%%%%%%%%%%%%%%%%%%%%%%%%%%%%%%%%%%%%%%%%%%%%
   %%%%%%%%%%%%%%%%%%
    %%%%%%%%%%%%%%%%%%%%%%%%%%%%%%%%%%%%%%%%%%%%%%
    
    
    
    
 
%    
%\tableofcontents
    %\addcontentsline{toc}{section}{Abstract}
    %\addcontentsline{toc}{section}{Acknowledgment}
    
    %%%%% Learn       %%%%%%%%%%%%%%%%%%%%%%%%%%%
    %%%%%%%%%%%%%%%%%%%%%%%%%%%%%%%%
    
    
    %%%%%%%%%%%%%%%%%%%%%%%%%%%%%%%%
    %%%%%%%%%%%%%%%%%%%%%%%%%%%%%%%%
    %%%%%%%%%%%%%%%%%%%%%%%%%%%%%%%%
    %\section{Useful URL's}
    %	\paragraph{FIRST TITLE}
    
    %		\hfill
    %\vspace{5mm} %5mm vertical space	
    
    %	Hi
    %	\paragraph{Para 1}
    %	\hfill
    %\vspace{5mm} %5mm vertical space
    %	https://earthsci.stanford.edu/computing/unix/formatting/latexexample.php
    %\vspace{5mm} %5mm vertical space
    %	http://mally.stanford.edu/\textbf{$\urcorner$}sr/computing/latex-example.html
    %\vspace{5mm} %5mm vertical space
    %	http://www.electronics.oulu.fi/latex/examples/example\_1/
    %\vspace{5mm} %5mm vertical space
    %	https://en.wikibooks.org/wiki/LaTeX/Sample\_LaTeX\_documents
    %	\vspace{5mm} %5mm vertical space
    %https://www.sharelatex.com/learn/Main\_Page
    %	\paragraph{Para 2 Ref Manuals}
    %\hfill
    %\vspace{5mm} %5mm vertical space
    %http://tug.org/texinfohtml/latex2e.html
    %\vspace{5mm} %5mm vertical space
    %https://www.tug.org/utilities/plain/cseq.html
    %\vspace{5mm} %5mm vertical space
    %https://stuff.mit.edu/afs/athena/contrib/tex-contrib/beamer/pgf-1.01/doc/generic/pgf/version-for-tex4ht/en/pgfmanual.html#pgfmanualse9.html
    %matrix
    %https://tex.stackexchange.com/questions/300109/simple-visualization-of-3d-matrix
    %cube
    %https://tex.stackexchange.com/questions/29877/need-help-creating-a-3d-cube-from-a-2d-set-of-nodes-in-tikz/29882#29882
    % quick ref
    %https://www.library.caltech.edu/sites/default/files/latex-quickguide.pdf
    %
    % special char
    %http://www.combinatorics.net/weblib/a.9-10/a9.html
    %http://tug.ctan.org/info/symbols/comprehensive/symbols-a4.pdf
    %
    
    %%%%%%%%%%%%%%%%%%%%%%%%%%%%%%
    %\pagebreak
    %\section{PNG}
    %	\paragraph{Para 3 PNG} 
    %	\vspace{5mm} %5mm vertical space
    %	Graphics
    %	
    %%"firstXLS".png}
    %%\includegraphics{"firstXLS".png}
    
    
    %%%%%%%%%%%%%%%%%%%%%%%%%%%%%%%%
    %%\pagebreak
    %%\begin{landscape}
    %%\end{landscape}
    
    %%%%%%%%%%%%%%%%%%%%%%%%%%%%%%%%
    %%%%%%%%%%%%%%%%%%%%%%%%%%%%%%%%
    %%%%%%%%%%%%%%%%%%%%%%%%%%%%%%%%
    %%%%%%%%%%%%%%%%%%%%%%%%%%%%%%%%
    %%%%%%%%%%%%%%%%%%%%%%%%%%%%%%%%
    %%%%%%%%%%%%%%%%%%%%%%%%%%%%%%%%
    %%%%%%%%%%%%%%%%%%%%%%%%%%%%%%%%
%\pagebreak
%\section{Why ?}
%\subsection{Why ?  - And A Reason}
%\par
%\tab A long , long time ago 






%%%%%%%%%%%%%%%%%%%%%%%%%%%%%%%%%%
\subsection{Results}
\par 
Here are the results matrix with the possible primes painted in yellow. 
%\\
%This is for values of `n' =  -1 , 0 , 1 , 2.  The lowest number is in the top left corner , and the highest is in the bottom right corner.	
\\
\begin{tikzpicture}[every node/.style={anchor=north east ,fill=white,minimum width=1cm,minimum height=5mm}]
\tiny
\matrix (mA) [draw,matrix of math nodes]
{
    \colorbox{yellow}{(61)}   &                   (66)    &  \colorbox{yellow}{(71)} &                   (76)  &                   (81)   &                    (86)   &  \\
 (62)    & \colorbox{yellow}{(67)}   &                    (72)  & \colorbox{yellow}{(77)} &                   (82)   &                    (87)   &  \\
 (63)    &                   (68)    &  \colorbox{yellow}{(73)} &                   (78)  & \colorbox{yellow}{(83)}  &                    (88)   &  \\
 (64)    &                   (69)    &                    (74)  & \colorbox{yellow}{(79)} &                   (84)   &  \colorbox{yellow}{(89)}  &  \\
 (65)    &                   (70)    &                    (75)  &                   (80)  &                   (85)   &                    (90)   &  \colorbox{green}{n = 2} \\
};
\matrix (mB) [draw,matrix of math nodes] at ($(mA.south west)+(3,-1)$)
{
     \colorbox{yellow}{(31)}   &                   (36)    & \colorbox{yellow}{(41)} &                   (46)  &                    (51)    &                   (56)    &  \\
  (32)    & \colorbox{yellow}{(37)}   &                   (42)  & \colorbox{yellow}{(47)} &                    (52)    &                   (57)    &  \\
  (33)    &                   (38)    & \colorbox{yellow}{(43)} &                   (48)  &  \colorbox{yellow}{(53)}   &                   (58)    &  \\
  (34)    &                   (39)    &                   (44)  & \colorbox{yellow}{(49)} &                    (54)    & \colorbox{yellow}{(59)}   &  \\
  (35)    &                   (40)    &                   (45)  &                   (50)  &                    (55)    &                   (60)    & \colorbox{green}{n = 1} \\
};  
\matrix (mC) [draw,matrix of math nodes] at ($(mB.south west)+(3,-1)$)
{
 \colorbox{yellow}{(1)}   &                   (6)    & (16)                    &  \colorbox{yellow}{(11)} & (21)                     &  (26)                     &  \\
(2)    & \colorbox{yellow}{(7)}   & \colorbox{yellow}{(17)} &  (12)                    & (22)                     &  (27)                     &  \\
(3)    &                   (8)    & (18)                    &  \colorbox{yellow}{(13)} & \colorbox{yellow}{(23)}  &  (28)                     &  \\
(4)    &                   (9)    & \colorbox{yellow}{(19)} &  (14)                    & (24)                     &  \colorbox{yellow}{(29)}  &  \\
(5)    &                  (10)    & (20)                    &  (15)                    & (25)                     &  (30)                     &  \colorbox{green}{n = 0} \\
};
\matrix (mD) [draw,matrix of math nodes] at ($(mC.south west)+(3,-1)$)	
{   
    \colorbox{yellow}{(-29)}  &                  (-24)   & \colorbox{yellow}{(-19)} &                   (-14)  &                    (-9)   &                    (-4)   &  \\
   (-28)   &\colorbox{yellow}{(-23)}  &                   (-18)  & \colorbox{yellow}{(-13)} &                    (-8)   &                    (-3)   &  \\
   (-27)   &                  (-22)   & \colorbox{yellow}{(-17)} &                   (-12)  &  \colorbox{yellow}{(-7)}  &                    (-2)   &  \\
   (-26)   &                  (-21)   &                   (-16)  & \colorbox{yellow}{(-11)} &                    (-6)   &  \colorbox{yellow}{(-1)}  &  \\
   (-25)   &                  (-20)   &                   (-15)  &                   (-10)  &                    (-5)   &                     (0)   &  \colorbox{green}{n = - 1} \\
};                                   



\draw[dashed](mA.north east)--(mD.north east);
\draw[dashed](mA.north west)--(mD.north west);
\draw[dashed](mA.south east)--(mD.south east);
\normalfont 
\end{tikzpicture}
\pagebreak

\subsection{3D View Of The Cash Pile}
\par 
This is a view of the Cash Pile , it is a \textbf{small portion} of the numbers. The bottom face is a 3x10 matrix of the formulas. The right hand face is  `n' with each row the same value. In the interior are the results. This is just a \textbf{pile} of numbers that goes on forever  ; down (\textendash $\infty$ ) or up (+ $\infty$).
\\
\\
\begin{tikzpicture}
{
    \tikzcuboid{%
        shiftx=4cm,%
        shifty=0cm,%
        scale=0.90,%
        rotation=-83,%
        densityx=1,%
        densityy=1,%
        densityz=2,%
        dimx=10,%
        dimy=5,%
        dimy=5,%
        linefront=red!75!black,%
        linetop=red!50!black,%
        lineright=red!25!black,%
        fillfront=red!25!white,%
        filltop=red!50!white,%
        fillright=red!75!white%
    }
}
\end{tikzpicture}


\subsection{Bad Par}
\par 
No Indent  - Why Not?
\\
\\
\par 
Second par - is indented
\pagebreak
\par
New Page good par indent



\end{document}